% Options for packages loaded elsewhere
\PassOptionsToPackage{unicode}{hyperref}
\PassOptionsToPackage{hyphens}{url}
%
\documentclass[
]{article}
\usepackage{amsmath,amssymb}
\usepackage{lmodern}
\usepackage{ifxetex,ifluatex}
\ifnum 0\ifxetex 1\fi\ifluatex 1\fi=0 % if pdftex
  \usepackage[T1]{fontenc}
  \usepackage[utf8]{inputenc}
  \usepackage{textcomp} % provide euro and other symbols
\else % if luatex or xetex
  \usepackage{unicode-math}
  \defaultfontfeatures{Scale=MatchLowercase}
  \defaultfontfeatures[\rmfamily]{Ligatures=TeX,Scale=1}
\fi
% Use upquote if available, for straight quotes in verbatim environments
\IfFileExists{upquote.sty}{\usepackage{upquote}}{}
\IfFileExists{microtype.sty}{% use microtype if available
  \usepackage[]{microtype}
  \UseMicrotypeSet[protrusion]{basicmath} % disable protrusion for tt fonts
}{}
\makeatletter
\@ifundefined{KOMAClassName}{% if non-KOMA class
  \IfFileExists{parskip.sty}{%
    \usepackage{parskip}
  }{% else
    \setlength{\parindent}{0pt}
    \setlength{\parskip}{6pt plus 2pt minus 1pt}}
}{% if KOMA class
  \KOMAoptions{parskip=half}}
\makeatother
\usepackage{xcolor}
\IfFileExists{xurl.sty}{\usepackage{xurl}}{} % add URL line breaks if available
\IfFileExists{bookmark.sty}{\usepackage{bookmark}}{\usepackage{hyperref}}
\hypersetup{
  pdftitle={Assignments},
  hidelinks,
  pdfcreator={LaTeX via pandoc}}
\urlstyle{same} % disable monospaced font for URLs
\usepackage[margin=1in]{geometry}
\usepackage{color}
\usepackage{fancyvrb}
\newcommand{\VerbBar}{|}
\newcommand{\VERB}{\Verb[commandchars=\\\{\}]}
\DefineVerbatimEnvironment{Highlighting}{Verbatim}{commandchars=\\\{\}}
% Add ',fontsize=\small' for more characters per line
\usepackage{framed}
\definecolor{shadecolor}{RGB}{248,248,248}
\newenvironment{Shaded}{\begin{snugshade}}{\end{snugshade}}
\newcommand{\AlertTok}[1]{\textcolor[rgb]{0.94,0.16,0.16}{#1}}
\newcommand{\AnnotationTok}[1]{\textcolor[rgb]{0.56,0.35,0.01}{\textbf{\textit{#1}}}}
\newcommand{\AttributeTok}[1]{\textcolor[rgb]{0.77,0.63,0.00}{#1}}
\newcommand{\BaseNTok}[1]{\textcolor[rgb]{0.00,0.00,0.81}{#1}}
\newcommand{\BuiltInTok}[1]{#1}
\newcommand{\CharTok}[1]{\textcolor[rgb]{0.31,0.60,0.02}{#1}}
\newcommand{\CommentTok}[1]{\textcolor[rgb]{0.56,0.35,0.01}{\textit{#1}}}
\newcommand{\CommentVarTok}[1]{\textcolor[rgb]{0.56,0.35,0.01}{\textbf{\textit{#1}}}}
\newcommand{\ConstantTok}[1]{\textcolor[rgb]{0.00,0.00,0.00}{#1}}
\newcommand{\ControlFlowTok}[1]{\textcolor[rgb]{0.13,0.29,0.53}{\textbf{#1}}}
\newcommand{\DataTypeTok}[1]{\textcolor[rgb]{0.13,0.29,0.53}{#1}}
\newcommand{\DecValTok}[1]{\textcolor[rgb]{0.00,0.00,0.81}{#1}}
\newcommand{\DocumentationTok}[1]{\textcolor[rgb]{0.56,0.35,0.01}{\textbf{\textit{#1}}}}
\newcommand{\ErrorTok}[1]{\textcolor[rgb]{0.64,0.00,0.00}{\textbf{#1}}}
\newcommand{\ExtensionTok}[1]{#1}
\newcommand{\FloatTok}[1]{\textcolor[rgb]{0.00,0.00,0.81}{#1}}
\newcommand{\FunctionTok}[1]{\textcolor[rgb]{0.00,0.00,0.00}{#1}}
\newcommand{\ImportTok}[1]{#1}
\newcommand{\InformationTok}[1]{\textcolor[rgb]{0.56,0.35,0.01}{\textbf{\textit{#1}}}}
\newcommand{\KeywordTok}[1]{\textcolor[rgb]{0.13,0.29,0.53}{\textbf{#1}}}
\newcommand{\NormalTok}[1]{#1}
\newcommand{\OperatorTok}[1]{\textcolor[rgb]{0.81,0.36,0.00}{\textbf{#1}}}
\newcommand{\OtherTok}[1]{\textcolor[rgb]{0.56,0.35,0.01}{#1}}
\newcommand{\PreprocessorTok}[1]{\textcolor[rgb]{0.56,0.35,0.01}{\textit{#1}}}
\newcommand{\RegionMarkerTok}[1]{#1}
\newcommand{\SpecialCharTok}[1]{\textcolor[rgb]{0.00,0.00,0.00}{#1}}
\newcommand{\SpecialStringTok}[1]{\textcolor[rgb]{0.31,0.60,0.02}{#1}}
\newcommand{\StringTok}[1]{\textcolor[rgb]{0.31,0.60,0.02}{#1}}
\newcommand{\VariableTok}[1]{\textcolor[rgb]{0.00,0.00,0.00}{#1}}
\newcommand{\VerbatimStringTok}[1]{\textcolor[rgb]{0.31,0.60,0.02}{#1}}
\newcommand{\WarningTok}[1]{\textcolor[rgb]{0.56,0.35,0.01}{\textbf{\textit{#1}}}}
\usepackage{graphicx}
\makeatletter
\def\maxwidth{\ifdim\Gin@nat@width>\linewidth\linewidth\else\Gin@nat@width\fi}
\def\maxheight{\ifdim\Gin@nat@height>\textheight\textheight\else\Gin@nat@height\fi}
\makeatother
% Scale images if necessary, so that they will not overflow the page
% margins by default, and it is still possible to overwrite the defaults
% using explicit options in \includegraphics[width, height, ...]{}
\setkeys{Gin}{width=\maxwidth,height=\maxheight,keepaspectratio}
% Set default figure placement to htbp
\makeatletter
\def\fps@figure{htbp}
\makeatother
\setlength{\emergencystretch}{3em} % prevent overfull lines
\providecommand{\tightlist}{%
  \setlength{\itemsep}{0pt}\setlength{\parskip}{0pt}}
\setcounter{secnumdepth}{-\maxdimen} % remove section numbering
\ifluatex
  \usepackage{selnolig}  % disable illegal ligatures
\fi

\title{Assignments}
\author{}
\date{\vspace{-2.5em}}

\begin{document}
\maketitle

This page will contain all the assignments you submit for the class.

\hypertarget{instructions-for-all-assignments}{%
\subsubsection{Instructions for all
assignments}\label{instructions-for-all-assignments}}

I want you to submit your assignment as a PDF, so I can keep a record of
what the code looked like that day. I also want you to include your
answers on your personal GitHub website. This will be good practice for
editing your website and it will help you produce something you can keep
after the class is over.

\begin{enumerate}
\def\labelenumi{\arabic{enumi}.}
\item
  Download the Assignment1.Rmd file from Canvas. You can use this as a
  template for writing your answers. It's the same as what you can see
  on my website in the Assignments tab. Once we're done with this I'll
  edit the text on the website to include the solutions.
\item
  On RStudio, open a new R script in RStudio (File \textgreater{} New
  File \textgreater{} R Script). This is where you can test out your R
  code. You'll write your R commands and draw plots here.
\item
  Once you have finalized your code, copy and paste your results into
  this template (Assignment 1.Rmd). For example, if you produced a plot
  as the solution to one of the problems, you can copy and paste the R
  code in R markdown by using the
  \texttt{\textasciigrave{}\textasciigrave{}\{r\}\ \textasciigrave{}\textasciigrave{}\textasciigrave{}}
  command. Answer the questions in full sentences and Save.
\item
  Produce a PDF file with your answers. To do this, knit to PDF (use
  Knit button at the top of RStudio), locate the PDF file in your docs
  folder (it's in the same folder as the Rproj), and submit that on on
  Canvas in Assignment 1.
\item
  Build Website, go to GitHub desktop, commit and push. Now your
  solutions should be on your website as well.
\end{enumerate}

\hypertarget{assignment-1}{%
\section{Assignment 1}\label{assignment-1}}

\textbf{Collaborators: Lorem Ipsum. }

This assignment is due on Canvas on Monday 9/20 before class, at 10:15
am. Include the name of anyone with whom you collaborated at the top of
the assignment.

\hypertarget{problem-1}{%
\subsubsection{Problem 1}\label{problem-1}}

Install the datasets package on the console below using
\texttt{install.packages("datasets")}. Now load the library.

Load the USArrests dataset and rename it \texttt{dat}. Note that this
dataset comes with R, in the package datasets, so there's no need to
load data from your computer. Why is it useful to rename the dataset?

\hypertarget{problem-2}{%
\subsubsection{Problem 2}\label{problem-2}}

Use this command to make the state names into a new variable called
State.

\begin{Shaded}
\begin{Highlighting}[]
\NormalTok{dat}\SpecialCharTok{$}\NormalTok{state }\OtherTok{\textless{}{-}} \FunctionTok{tolower}\NormalTok{(}\FunctionTok{rownames}\NormalTok{(USArrests))}
\end{Highlighting}
\end{Shaded}

This dataset has the state names as row names, so we just want to make
them into a new variable. We also make them all lower case, because that
will help us draw a map later - the map function requires the states to
be lower case.

List the variables contained in the dataset \texttt{USArrests}.

\hypertarget{problem-3}{%
\subsubsection{Problem 3}\label{problem-3}}

What type of variable (from the DVB chapter) is \texttt{Murder}?

Answer:

What R Type of variable is it?

Answer:

\hypertarget{problem-4}{%
\subsubsection{Problem 4}\label{problem-4}}

What information is contained in this dataset, in general? What do the
numbers mean?

Answer:

\hypertarget{problem-5}{%
\subsubsection{Problem 5}\label{problem-5}}

Draw a histogram of \texttt{Murder} with proper labels and title.

\hypertarget{problem-6}{%
\subsubsection{Problem 6}\label{problem-6}}

Please summarize \texttt{Murder} quantitatively. What are its mean and
median? What is the difference between mean and median? What is a
quartile, and why do you think R gives you the 1st Qu. and 3rd Qu.?

\hypertarget{problem-7}{%
\subsubsection{Problem 7}\label{problem-7}}

Repeat the same steps you followed for \texttt{Murder}, for the
variables \texttt{Assault} and \texttt{Rape}. Now plot all three
histograms together. You can do this by using the command
\texttt{par(mfrow=c(3,1))} and then plotting each of the three.

What does the command par do, in your own words (you can look this up by
asking R \texttt{?par})?

Answer:

What can you learn from plotting the histograms together?

Answer:

\hypertarget{problem-8}{%
\subsubsection{Problem 8}\label{problem-8}}

In the console below (not in text), type
\texttt{install.packages("maps")} and press Enter, and then type
\texttt{install.packages("ggplot2")} and press Enter. This will install
the packages so you can load the libraries.

Run this code:

\begin{Shaded}
\begin{Highlighting}[]
\FunctionTok{library}\NormalTok{(}\StringTok{\textquotesingle{}maps\textquotesingle{}}\NormalTok{) }
\FunctionTok{library}\NormalTok{(}\StringTok{\textquotesingle{}ggplot2\textquotesingle{}}\NormalTok{) }

\FunctionTok{ggplot}\NormalTok{(dat, }\FunctionTok{aes}\NormalTok{(}\AttributeTok{map\_id=}\NormalTok{state, }\AttributeTok{fill=}\NormalTok{Murder)) }\SpecialCharTok{+} 
  \FunctionTok{geom\_map}\NormalTok{(}\AttributeTok{map=}\FunctionTok{map\_data}\NormalTok{(}\StringTok{"state"}\NormalTok{)) }\SpecialCharTok{+} 
  \FunctionTok{expand\_limits}\NormalTok{(}\AttributeTok{x=}\FunctionTok{map\_data}\NormalTok{(}\StringTok{"state"}\NormalTok{)}\SpecialCharTok{$}\NormalTok{long, }\AttributeTok{y=}\FunctionTok{map\_data}\NormalTok{(}\StringTok{"state"}\NormalTok{)}\SpecialCharTok{$}\NormalTok{lat)}
\end{Highlighting}
\end{Shaded}

What does this code do? Explain what each line is doing.

Answer:

\[\\[2in]\]

\hypertarget{assignment-2}{%
\section{Assignment 2}\label{assignment-2}}

(Coming soon)

\end{document}
